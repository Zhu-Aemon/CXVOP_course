\documentclass[12pt, a4paper, oneside]{ctexart}
\usepackage{amsmath, amsthm, amssymb, bm, color, framed, graphicx, hyperref, mathrsfs, geometry, booktabs, threeparttable, multirow, diagbox, array, placeins, enumerate, mathtools}

\title{\textbf{最优化方法第二次作业}}
\author{王天柱}
\date{\today}
\linespread{1.1}
\geometry{a4paper,left=2cm,right=2cm,top=2cm,bottom=2cm}
\definecolor{shadecolor}{RGB}{241, 241, 255}
\newcounter{problemname}
\newenvironment{problem}{\begin{shaded}\stepcounter{problemname}\par\noindent\textbf{Problem\arabic{problemname}. }}{\end{shaded}\par}
\newenvironment{solution}{\par\noindent\textbf{Solution:}}{\par}
\newenvironment{note}{\par\noindent\textbf{题目\arabic{problemname}的注记. }}{\par}

\begin{document}

\maketitle

\begin{problem}
    \textbf{带$l_{2}$惩罚的部分优化问题}

    考虑问题
    \begin{equation} \label{con:target}
    \min_{\beta, \sigma \geq 0} f(\beta) + \frac{\lambda}{2}\sum_{i=1}^{n} g(\beta_{i}, \sigma_{i})
    \end{equation}
    其中$f$为定义在$\mathbb{R}^{n}$上的凸函数,$\lambda \geq 0$,且

    \begin{equation} \nonumber
        g(x, y) = \left\{
                \begin{array}{ll}  % 这里使用ll表示左对齐
                    \dfrac{x^2}{y} + y \qquad \mbox{if y > 0} \\
                    0 \qquad \mbox{if x=0, y=0} \\
                    \infty \qquad \mbox{else.}
                \end{array}
            \right.
    \end{equation}
    \begin{enumerate}[1.]
        \item 证明$g$是凸函数,即上述问题是凸优化问题
        \item 证明:$$\min_{y \geq 0} g(x, y) = 2 \lvert x \rvert$$
        \item 证明\eqref{con:target}中对于$\sigma \geq 0$的优化可得$l_{1}$惩罚问题:$$\min_{\beta} f(\beta) + \lambda \lVert \beta \rVert _{1}$$
    \end{enumerate}
\end{problem}

\begin{solution}

\end{solution}

\end{document}